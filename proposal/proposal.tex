\documentclass[12pt, a4paper]{article}
\usepackage[tmargin=1.0cm, bmargin=2.5cm, lmargin=2cm, rmargin=2cm]{geometry}
\usepackage[utf8]{inputenc}\DeclareUnicodeCharacter{2212}{-}
\usepackage[T1]{fontenc}
\usepackage{float}
\usepackage{lmodern}
\usepackage{hyperref}
\hypersetup{
colorlinks = true,
linkcolor  = black,
citecolor = black
}
% Bib stuff
\usepackage[
    backend=biber,
    style=apa,
]{biblatex}
\setlength{\bibhang}{0pt}
\setlength{\bibitemsep}{6pt}
\addbibresource[]{../peak_water.bib}
\usepackage{amsmath}
\usepackage{amssymb}
\usepackage{gensymb}
\usepackage{upgreek}
\usepackage{enumitem}
\usepackage{graphicx}
\usepackage{subcaption}
\graphicspath{{../plots/}}
\usepackage{xcolor, colortbl}
% Setup for the captions
\usepackage[hypcap=true,font={it}]{caption}
\captionsetup{belowskip=2pt, aboveskip=2pt}
\author{Erik Holmgren \\ Advisor: Fabien Maussion}
\title{Master thesis proposal: Peak water using the Open Global Glacier Model}
\date{March 2021}
\begin{document}
\maketitle
\noindent
\section{Motivation}
Mass loss from glaciers has increased during the second half of the 20th century
\parencite{vaughanObservationsCryosphere2013} and is prediced, in all current
climate projections, to continue throughout the 21st century
\parencite{ipccClimateChange20142014}. The magnitude of the end of century
glacial mass loss varies greatly depending on the region and climate scenario --
\textcite{hussNewModelGlobal2015} found a global glacier volume decrease between
25\% (RCP2.6) and 48\% (RCP8.5) and regional losses between 20 and 90\%.

% Simulating global glaciers \textcite{hussNewModelGlobal2015} found that the
% global glacier volume will decrease with 25$\pm$5\% for RCP2.6, 33$\pm$8\% for
% RCP4.5 and 48$\pm$9\% by the end of the century. Furthermore, they found large
% regional differences -- glaciers in central Europe and in low latitudes may lose
% as much as 90\% of their mass while glaciers in Arctic Canada and
% Antarctica/Subantarcitca lose 20\%. 

% End of century glacier mass exhibit a substantial spread depending on the GCM
% used for the emission scenario in the simulation. For instance, mass losses in
% Svalbard varies between 12\% and 90\% for RCP4.5
% \parencite{hussNewModelGlobal2015}.

Glaciers play an important role as storage magasins, delaying up to 79\% of the
total precipitation falling on the glacier surface (Aral Basin) as meltwater
runoff later in the season. The benefits of this seasonal delay is particularly
important in regions with a warm and dry ablation season
\parencite{kaserContributionPotentialGlaciers2010}. One of these areas is the
Indus basin where, during the pre-monsoon season, up to 60\% of the total
irrigation volume comes from either snow or glacier melt -- resulting in an 11\%
increase of the total crop production
\parencite{biemansImportanceSnowGlacier2019}. The Indus basin is also an example
of large river basins which under the present climate experiences water scarcity
-- threatening the food security for millions of people
\parencite{kummuClimatedrivenInterannualVariability2014} in an area where large
amounts of the freshwater resource is shared across state borders where the risk
for amred conflict is high \parencite{schleussnerArmedconflictRisksEnhanced2016,
pritchardAsiaShrinkingGlaciers2019}. The populated areas on the dry, western,
slopes of the Andes are other examples of regions depending on glacier meltwater
for potable water and power generation.
\textcite{vergaraEconomicImpactsRapid2007} estimate the cost of mitigation and
adaption to retreating glaciers in the Andes from US\$300 million up to US\$
1.5 billion.
% of glacier melt include significant contributions to sea level rise (e.g.
% \cite{marzeionFutureSealevelChange2012a})

% More direct societal impacts The Indus basin is experiencing water scarcity under the present climate
% \parencite{kummuClimatedrivenInterannualVariability2014}.

% Out of the total precipitation falling onto the glacier surface, between 79 and
% 17\% will experience a seasonal delay -- meaning that the water will be stored
% in the glacier and released later in the season. The relative importance of
% glacier melt water to the basin water availability decreases in the presence of
% liquid precipitation, hence the positive effects of seasonally delayed water
% release from glaciers are more pronounced in areas experiencing warm and dry
% ablation seasons \parencite{kaserContributionPotentialGlaciers2010}.

% The monthly percentage of total water input to a basin that experience seasonal
% delayed water release by glaciers decreases downstream the river while the
% population generally increases. Thus, the societal impacts of seasonal delayed
% water release by glacier melt reaches a maximum at intermediate altitude bands
% \parencite{kaserContributionPotentialGlaciers2010}.

\section{State of the art -- Glacial hydrology}
In alpine catchments snow accumulation and snow melt are the main contributors
to runoff generation, while the influence of precipitation is small, during the
months between June and October \parencite{zappaSeasonalWaterBalance2003}. 

The main factors controlling the discharge hydrograph are the topographical
structure, the seasonal air temperature gradient, and the seasonal distribution
of precipitation \parencite{zappaSeasonalWaterBalance2003}.

Runoff peaks later in the summer -- when the melt line has crept further up and
thus exposing more of the catchment area to melt \parencite{zappaSeasonalWaterBalance2003}.

\textcite{kaserContributionPotentialGlaciers2010} estimated the societal
importance of glacier melt water under the assumption that the glaciers were in
equilibrium with the local climate -- any runoff from glacier net mass loss was
not included in their analysis. \textcite{blissGlobalResponseGlacier2014} showed
that glacier net mass loss is an important part of the total glacier runoff,
indicating that the societal importance of glacier melt water might be higher
than the estimates from \textcite{kaserContributionPotentialGlaciers2010}.

The ratio between the summer runoff and total runoff of a glacier rises strongly
with the glaciated area \parencite{janssonConceptGlacierStorage2003}.

\section{Peak water using the OGGM}
State of the art peak water estimations
\parencite{rounceGlacierMassChange2020,hussGlobalscaleHydrologicalResponse2018}
have relied on parametrizing the re-distribution of mass throughout the glacier
with so called mass re-distribution curves developed by
\textcite{hussFutureHighmountainHydrology2010}. The parameterization by
\textcite{hussFutureHighmountainHydrology2010} is a clear step up in
performance compared to previous ice flow parametrizations but still relies on
known glacier measurements for calibration. The flow of non-measured glaciers
is estimated from known glaciers of the same size. 

Employing the Open Global Glacier Model (OGGM,
\cite{maussionOpenGlobalGlacier2019}) for peak water calculations would be the
first time a physical ice flow model is applied globally to calculate glacier
runoff. It would be a step towards mitigating the problem of over
parametrization present in the current global glacier models used for
hydrological analysis. A new set of runoff estimates, based on a different
modelling framework, will also broaden the scientific background about the
subject. The OGGM uses the same mass balance scheme, a degree day model, as for
example PyGEM (used by \cite{rounceGlacierMassChange2020}). Thus, any
differences in the annual mass balance should stem from the different
implementations of ice dynamics -- possibly leading to slightly different
area/length estimations and thus a different runoff.

In its current state OGGM does not save any hydrological outputs (not true any
more...) so before any runoff calculations can be made this has to be added to
the model and tested. The common approach is to use a fixed gauge -- a
hypothetical measuring station at the terminus of the glacier, measuring all
water leaving the initially glaciated area. This implies calculating the runoff,
$Q$, from glacial melt $\alpha$, and liquid precipitation $p_{liquid}$:
\begin{equation}
    Q = p_{liquid} + \alpha - R,
\end{equation}
where $R$ is the refreezing of melt water within the glacier. The runoff from
snow melt and liquid precipitation is calculated from the initially glaciated
area and is, as the glaciated area shrinks, divided into two part: on glacier
and off glacier runoff. 



\printbibliography
\end{document}
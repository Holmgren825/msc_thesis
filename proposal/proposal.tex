\documentclass[12pt, a4paper]{article}
\usepackage[tmargin=1.0cm, bmargin=2.5cm, lmargin=2cm, rmargin=2cm]{geometry}
\usepackage[utf8]{inputenc}\DeclareUnicodeCharacter{2212}{-}
\usepackage[T1]{fontenc}
\usepackage{float}
\usepackage{lmodern}
\usepackage{hyperref}
\hypersetup{
colorlinks = true,
linkcolor  = black,
citecolor = black
}
% Bib stuff
\usepackage[
    backend=biber,
    style=apa,
]{biblatex}
\setlength{\bibhang}{0pt}
\setlength{\bibitemsep}{6pt}
\addbibresource[]{../peak_water.bib}
\usepackage{amsmath}
\usepackage{amssymb}
\usepackage{gensymb}
\usepackage{upgreek}
\usepackage{enumitem}
\usepackage{graphicx}
\usepackage{subcaption}
\graphicspath{{../plots/}}
\usepackage{xcolor, colortbl}
% Setup for the captions
\usepackage[hypcap=true,font={it}]{caption}
\captionsetup{belowskip=2pt, aboveskip=2pt}
\author{Erik Holmgren \\ Advisor: Fabien Maussion}
\title{Master thesis proposal: Peak water using the Open Global Glacier Model}
\date{March 2021}
\begin{document}
\maketitle
\noindent
\section{Motivation}
Glacier mass loss has increased during the second half of the 20th century
\parencite{vaughanObservationsCryosphere2013} and is predicted, in all current
climate projections, to continue throughout the 21st century
\parencite{ipccClimateChange20142014}. The magnitude of the end of century
glacial mass loss varies greatly depending on the region and climate scenario --
\textcite{hussNewModelGlobal2015} found a global glacier volume decrease between
25\% (RCP2.6) and 48\% (RCP8.5) and regional losses varying between 20 and 90\%.

% Simulating global glaciers \textcite{hussNewModelGlobal2015} found that the
% global glacier volume will decrease with 25$\pm$5\% for RCP2.6, 33$\pm$8\% for
% RCP4.5 and 48$\pm$9\% by the end of the century. Furthermore, they found large
% regional differences -- glaciers in central Europe and in low latitudes may lose
% as much as 90\% of their mass while glaciers in Arctic Canada and
% Antarctica/Subantarcitca lose 20\%. 

% End of century glacier mass exhibit a substantial spread depending on the GCM
% used for the emission scenario in the simulation. For instance, mass losses in
% Svalbard varies between 12\% and 90\% for RCP4.5
% \parencite{hussNewModelGlobal2015}.

Glaciers play an important role as a form of water storage, delaying up to 79\%
of the total precipitation falling on the glacier surface (Aral Basin), through
the release of meltwater during the ablation season. The benefits of this
seasonal delay is particularly important in regions with a warm and dry ablation
season \parencite{kaserContributionPotentialGlaciers2010}. One of those areas is
the Indus basin where, during the pre-monsoon season, up to 60\% of the total
irrigation volume comes from either snow or glacier melt -- with a resulting an
11\% increase of the total crop production
\parencite{biemansImportanceSnowGlacier2019}. The Indus basin is also an example
of large river basins which under the present climate experiences water scarcity
-- threatening the food security for millions of people
\parencite{kummuClimatedrivenInterannualVariability2014}. This in an area where
large amounts of the freshwater resource is shared across state borders where
the risk for armed conflict is high
\parencite{schleussnerArmedconflictRisksEnhanced2016,
pritchardAsiaShrinkingGlaciers2019}. 

The populated areas on the dry, western, slopes of the Andes are other examples
of regions depending on glacier meltwater for potable water and power
generation. \textcite{vergaraEconomicImpactsRapid2007} estimate the cost of
mitigation and adaption to retreating glaciers in the Andes to between US\$300
million and US\$ 1.5 billion.
% of glacier melt include significant contributions to sea level rise (e.g.
% \cite{marzeionFutureSealevelChange2012a})

% More direct societal impacts The Indus basin is experiencing water scarcity under the present climate
% \parencite{kummuClimatedrivenInterannualVariability2014}.

% Out of the total precipitation falling onto the glacier surface, between 79 and
% 17\% will experience a seasonal delay -- meaning that the water will be stored
% in the glacier and released later in the season. The relative importance of
% glacier melt water to the basin water availability decreases in the presence of
% liquid precipitation, hence the positive effects of seasonally delayed water
% release from glaciers are more pronounced in areas experiencing warm and dry
% ablation seasons \parencite{kaserContributionPotentialGlaciers2010}.

% The monthly percentage of total water input to a basin that experience seasonal
% delayed water release by glaciers decreases downstream the river while the
% population generally increases. Thus, the societal impacts of seasonal delayed
% water release by glacier melt reaches a maximum at intermediate altitude bands
% \parencite{kaserContributionPotentialGlaciers2010}.

\section{State of the art -- Glacial hydrology}
Glaciers store water in multiple ways -- as a liquid in surface snow and firn,
in crevasses, drainage networks, englacial pockets and surface pools. Or as s
solid as snow, firn and ice \parencite{janssonConceptGlacierStorage2003}. The
main factors controlling the discharge hydrograph of an Alpine basin is the
topographical structure, the seasonal air temperature gradient, the seasonal
distribution of precipitation \parencite{zappaSeasonalWaterBalance2003}, and the
percentage of glaciated area \parencite{janssonConceptGlacierStorage2003}. Melt
is the main contributing process to annual glacier runoff generation
\parencite{zappaSeasonalWaterBalance2003}. Thus, the ratio of summer runoff to
the annual runoff will increase with an increasing glaciated area
\parencite{chenInfluenceAlpineGlaciers1990}.

% Glacier runoff peaks late in the summer -- when the melt line has crept further
% up and thus exposing more of the catchment area to melt
% \parencite{zappaSeasonalWaterBalance2003}.

% In alpine catchments snow accumulation and snow melt are the main contributors
% to runoff generation, while the influence of precipitation is small, during the
% months between June and October \parencite{zappaSeasonalWaterBalance2003}. 

The estimated the societal importance of glacier melt water from
\textcite{kaserContributionPotentialGlaciers2010} was made under the assumption
that the glaciers were in equilibrium with the local climate -- i.e. none of the
runoff estimations included any net mass loss.
\textcite{blissGlobalResponseGlacier2014} showed that glacier net mass loss is
an important part of the total glacier runoff, indicating that the societal
importance of glacier melt water might be higher than the estimates from
\textcite{kaserContributionPotentialGlaciers2010}.

This is where I introduce peak water.

\section{Peak water using the OGGM}
State of the art peak water estimations
\parencite{rounceGlacierMassChange2020,hussGlobalscaleHydrologicalResponse2018}
have relied on parametrizing the re-distribution of mass throughout the glacier
with so called mass re-distribution curve, developed by
\textcite{hussFutureHighmountainHydrology2010}. This parameterization is a clear
step up in performance compared to previous ice flow parametrizations, but still
relies on glacier DEMs for calibration. The flow of non-measured glaciers will
be estimated from known glaciers of a similar size. 

Employing the Open Global Glacier Model (OGGM,
\cite{maussionOpenGlobalGlacier2019}) for peak water calculations would be the
first time a physical ice flow model is applied globally to calculate glacier
runoff. It would be a step towards mitigating the problem of over
parametrization present in the current global glacier models used for
hydrological analysis. Including ice dynamics in global glacier simulations
result in reduced ice losses compared to parametrized models
\parencite{zekollariModellingFutureEvolution2019}. It also allow the glacier to
not only shrink, a limitation of parametrized ice dynamics, but also to grow. Using the OGGM will thus provide a new view of global runoff and peak water
estimations, expanding the comprehension of the subject.

% A new set of runoff estimates, based on a different modelling framework, will
% broaden the scientific background about the subject. The OGGM uses the same
% mass balance scheme, a degree day model, as for example PyGEM (used by
% \cite{rounceGlacierMassChange2020}). Thus, any differences in the annual mass
% balance should stem from the different implementations of ice dynamics --
% possibly leading to slightly different area/length estimations and thus a
% different runoff.

In its current state OGGM does not save any hydrological outputs (not true any
more...) so before any runoff calculations can be made this has to be added to
the model. The common approach is to use a so called fixed gauge -- a
hypothetical measuring station at the terminus of the glacier, measuring all
water leaving the initially glaciated area. 

The runoff, $Q$, is calculated from the glacial melt $\alpha$, the liquid
precipitation $p_{liquid}$, and the refreezing of meltwater within the glacier
$R$ as:
\begin{equation}
    Q = \alpha + p_{liquid} - R.
\end{equation}
The runoff coming from snow melt and liquid precipitation is calculated on the
initially glaciated area and is, as the glaciated area shrinks, divided into two
parts: on glacier and off glacier runoff. 


\subsection{Research questions}
For this thesis I will try to answer the following questions:
\begin{enumerate}
    \item \textbf{How does the inclusion of ice dynamics in a global glacier
    model change the future temporal and spatial variation of peak water?}
    The inclusion of ice dynamics will results in a different annual mass
    balance compared to models relying on parametrization. Since the runoff
    estimations are based on the mass balance -- any changes to its calculation
    should result in a different estimate of peak water.
    \item \textbf{High mountain Asia, when will the basins most dependent on
    glacier runoff reach peak water?} This would basically be done to
    corroborate on the previous studies that have been done. The \textbf{Indus
    basin}.
    \item \textbf{At which levels, and during what time, will runoff levels
    begin to stabilise again?} Peak water gives a measure of when the annual
    runoff from glaciers reaches a maximum, but what about the long term
    equilibrium? What will the future water supply look like?
    \item \textbf{How will the seasonal hydrograph change for future runoff
    projections?} Will glaciers release water earlier in the season? Or later?
    Also connects to the previous question -- how is the annual runoff affected?
    
\end{enumerate}


\printbibliography
\end{document}